% Options for packages loaded elsewhere
\PassOptionsToPackage{unicode}{hyperref}
\PassOptionsToPackage{hyphens}{url}
%
\documentclass[
]{article}
\usepackage{amsmath,amssymb}
\usepackage{iftex}
\ifPDFTeX
  \usepackage[T1]{fontenc}
  \usepackage[utf8]{inputenc}
  \usepackage{textcomp} % provide euro and other symbols
\else % if luatex or xetex
  \usepackage{unicode-math} % this also loads fontspec
  \defaultfontfeatures{Scale=MatchLowercase}
  \defaultfontfeatures[\rmfamily]{Ligatures=TeX,Scale=1}
\fi
\usepackage{lmodern}
\ifPDFTeX\else
  % xetex/luatex font selection
\fi
% Use upquote if available, for straight quotes in verbatim environments
\IfFileExists{upquote.sty}{\usepackage{upquote}}{}
\IfFileExists{microtype.sty}{% use microtype if available
  \usepackage[]{microtype}
  \UseMicrotypeSet[protrusion]{basicmath} % disable protrusion for tt fonts
}{}
\makeatletter
\@ifundefined{KOMAClassName}{% if non-KOMA class
  \IfFileExists{parskip.sty}{%
    \usepackage{parskip}
  }{% else
    \setlength{\parindent}{0pt}
    \setlength{\parskip}{6pt plus 2pt minus 1pt}}
}{% if KOMA class
  \KOMAoptions{parskip=half}}
\makeatother
\usepackage{xcolor}
\usepackage[margin=1in]{geometry}
\usepackage{graphicx}
\makeatletter
\def\maxwidth{\ifdim\Gin@nat@width>\linewidth\linewidth\else\Gin@nat@width\fi}
\def\maxheight{\ifdim\Gin@nat@height>\textheight\textheight\else\Gin@nat@height\fi}
\makeatother
% Scale images if necessary, so that they will not overflow the page
% margins by default, and it is still possible to overwrite the defaults
% using explicit options in \includegraphics[width, height, ...]{}
\setkeys{Gin}{width=\maxwidth,height=\maxheight,keepaspectratio}
% Set default figure placement to htbp
\makeatletter
\def\fps@figure{htbp}
\makeatother
\setlength{\emergencystretch}{3em} % prevent overfull lines
\providecommand{\tightlist}{%
  \setlength{\itemsep}{0pt}\setlength{\parskip}{0pt}}
\setcounter{secnumdepth}{5}
\usepackage{titling}
\pretitle{\begin{center} \includegraphics[width=2in,height=2in]{ufsj.png}\LARGE\\}
\posttitle{\end{center}}
\ifLuaTeX
  \usepackage{selnolig}  % disable illegal ligatures
\fi
\usepackage{bookmark}
\IfFileExists{xurl.sty}{\usepackage{xurl}}{} % add URL line breaks if available
\urlstyle{same}
\hypersetup{
  pdftitle={Relatório 06},
  pdfauthor={Gustavo Cardoso},
  hidelinks,
  pdfcreator={LaTeX via pandoc}}

\title{Relatório 06}
\author{Gustavo Cardoso}
\date{22/05/2025}

\begin{document}
\maketitle

\renewcommand*\contentsname{Sumário}
{
\setcounter{tocdepth}{2}
\tableofcontents
}
\begin{center}\rule{0.5\linewidth}{0.5pt}\end{center}

\section{Objetivo}\label{objetivo}

Este trabalho tem como objetivo estudar os procedimentos de comparações
múltiplas, com foco no Teste SNK. Para isso, buscamos:

\begin{itemize}
\item
  Investigar quem desenvolveu o teste;
\item
  Identificar as situações ideais para sua aplicação;
\item
  Demonstrar sua aplicação utilizando o software R;
\item
  Apresentar os procedimentos analíticos e cálculos à mão.
\end{itemize}

\section{Apresentação do
relatório}\label{apresentauxe7uxe3o-do-relatuxf3rio}

O presente relatório tem como finalidade apresentar um estudo detalhado
sobre o teste de comparações múltiplas conhecido como SNK
(Student-Newman-Keuls). A pesquisa contempla aspectos históricos,
fundamentos teóricos, aplicação prática em dados simulados no R
(voltados à Engenharia Civil) e, por fim, o detalhamento dos cálculos
manuais que embasam o teste.

\section{Origem e desenvolvimento do Teste
SNK}\label{origem-e-desenvolvimento-do-teste-snk}

O Teste SNK é uma evolução do teste t de Student, criado por William
Sealy Gosset, que publicava sob o pseudônimo ``Student''. Em 1939,
Milton Newman introduziu um procedimento passo-a-passo para comparar
médias de forma ordenada. Posteriormente, em 1952, Maurice Keuls refinou
esse método, levando à formulação do teste que hoje conhecemos como
Student-Newman-Keuls (SNK).

Esse teste foi desenvolvido com o objetivo de detectar diferenças
significativas entre médias de tratamentos sem aumentar
consideravelmente a probabilidade de erro do tipo I, como ocorre com
múltiplas aplicações do teste t. Assim, o SNK equilibra sensibilidade e
controle de erro.

\section{Quando utilizar o Teste SNK}\label{quando-utilizar-o-teste-snk}

O Teste SNK é indicado para ser utilizado após uma ANOVA significativa,
ou seja, quando o teste F indica que ao menos uma média difere das
demais. Ele é apropriado para:

\begin{itemize}
\item
  Comparar médias de tratamentos em experimentos com 3 ou mais níveis;
\item
  Delineamentos simples como DIC (Delineamento Inteiramente Casualizado)
  ou DBC (Delineamento em Blocos Casualizados);
\item
  Situações em que se deseja maior sensibilidade em comparações
  ordenadas.
\end{itemize}

\section{Exemplo no R}\label{exemplo-no-r}

\begin{verbatim}
## Warning: pacote 'agricolae' foi compilado no R versão 4.4.3
\end{verbatim}

\begin{verbatim}
##             Df Sum Sq Mean Sq F value   Pr(>F)    
## tratamento   3 125.82   41.94   23.29 4.43e-06 ***
## Residuals   16  28.81    1.80                     
## ---
## Signif. codes:  0 '***' 0.001 '**' 0.01 '*' 0.05 '.' 0.1 ' ' 1
\end{verbatim}

\begin{verbatim}
## $statistics
##    MSerror Df    Mean       CV
##   1.800765 16 31.4218 4.270683
## 
## $parameters
##   test     name.t ntr alpha
##    SNK tratamento   4  0.05
## 
## $snk
##      Table CriticalRange
## 2 2.997999      1.799181
## 3 3.649139      2.189948
## 4 4.046093      2.428171
## 
## $means
##          resistencia       std r        se      Min      Max      Q25      Q50
## Aditivo1    28.29036 1.2165328 5 0.6001274 27.15929 30.33806 27.65473 28.10576
## Aditivo2    31.94682 1.3961875 5 0.6001274 30.48193 34.05808 31.17578 31.46521
## Aditivo3    30.30790 0.6394632 5 0.6001274 29.44416 31.22408 30.11068 30.35981
## Aditivo4    35.14214 1.8343537 5 0.6001274 32.44340 37.32299 34.38537 35.64721
##               Q75
## Aditivo1 28.19393
## Aditivo2 32.55310
## Aditivo3 30.40077
## Aditivo4 35.91176
## 
## $comparison
## NULL
## 
## $groups
##          resistencia groups
## Aditivo4    35.14214      a
## Aditivo2    31.94682      b
## Aditivo3    30.30790      b
## Aditivo1    28.29036      c
## 
## attr(,"class")
## [1] "group"
\end{verbatim}

Esse código aplica o teste SNK no R utilizando o pacote ``agricolae'',
agrupando os tratamentos por letras conforme suas diferenças
estatísticas.

\section{Cálculos Manuais do Teste
SNK}\label{cuxe1lculos-manuais-do-teste-snk}

Vamos supor as seguintes médias ordenadas:

Tratamento

Média (MPa)

A - 28.0

B - 30.0

C - 32.0

D - 35.0

Suponha que temos 5 repetições por tratamento e o QMResíduo (Erro médio
quadrático) obtido pela ANOVA foi 1.44.

Fórmula do erro padrão:

\begin{verbatim}
## SE = √(QMresíduo / n)
\end{verbatim}

Substituindo os valores:

\begin{verbatim}
## SE = √(1.44 / 5) = √0.288 = 0.5366
\end{verbatim}

Código em R:

\begin{verbatim}
## [1] 0.5366563
\end{verbatim}

Agora, calculamos os valores críticos q de Tukey para diferentes
distâncias (usando k = 4 tratamentos, gl = 16, alfa = 0.05):

\begin{verbatim}
## [1] 2.997999 3.649139 4.046093
\end{verbatim}

Cálculo do DMS para cada distância (r):

\begin{verbatim}
##      r=2      r=3      r=4 
## 1.608895 1.958333 2.171361
\end{verbatim}

Verificação das diferenças entre médias:

Diferenças:

\begin{itemize}
\tightlist
\item
  D vs A: 35 - 28 = 7.0 (r = 4)
\item
  D vs B: 35 - 30 = 5.0 (r = 3)
\item
  D vs C: 35 - 32 = 3.0 (r = 2)
\item
  C vs A: 32 - 28 = 4.0 (r = 3)
\item
  B vs A: 30 - 28 = 2.0 (r = 2)
\end{itemize}

Comparamos cada diferença com seu respectivo DMS. Se for maior, é
significativa.

\begin{verbatim}
##   Comparação Diferença   DMS     Resultado
## 1  D-A (r=4)         7 2.171 Significativa
## 2  D-B (r=3)         5 1.958 Significativa
## 3  D-C (r=2)         3 1.609 Significativa
## 4  C-A (r=3)         4 1.958 Significativa
## 5  B-A (r=2)         2 1.609 Significativa
\end{verbatim}

\section{Conclusão}\label{conclusuxe3o}

O Teste SNK é uma alternativa interessante para comparações múltiplas
após a ANOVA, oferecendo um bom balanço entre controle do erro do tipo I
e poder de detecção de diferenças reais. É especialmente útil em
experimentos com tratamentos ordenados, como nesse exemplo com
diferentes aditivos de concreto.

\end{document}
